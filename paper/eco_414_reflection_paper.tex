% Preamble
\documentclass[11pt, letterpaper]{article}

\usepackage[margin=1.5in]{geometry}
\usepackage{newtxtext}
\usepackage{newtxmath}
\usepackage{booktabs}
\usepackage{threeparttable}
\usepackage{tabularx}
\usepackage{array}
\usepackage{float}
\usepackage{optidef}

\title{Mathematical Economics Reflection Paper}
\author{Joe White}
\date{\today}

\begin{document}
\maketitle

\section{Introduction} 

This reflection examines \emph{The Effect of Ownership Structure on Prices in Geographically Differentiated Industries} by Raphael Thomadsen (2005), published in the \emph{RAND Journal of Economics}. The paper studies how ownership structure and geographic differentiation shape pricing in retail markets. Thomadsen presents results using two complementary methodologies. First, he reports regression estimates that document a relationship between ownership, geography, and prices. Second, he develops a structural model of demand and supply that incorporates the geographic layout of outlets. Using this structural model, he conducts counterfactual experiments to study how different ownership structures and spatial configurations affect equilibrium prices.

Thomadsen finds that mergers between franchisees can significantly increase prices. In the data, some outlets charge substantially higher prices than they would have under independent ownership. The price effects are strongest when commonly owned outlets are located close to one another, because the owner internalizes competition between nearby stores. However, the study also finds that even outlets located far enough apart that they would not directly influence each other’s pricing under separate ownership still experience price increases after a merger. This shows that the impact of a merger extends beyond immediate geographic proximity. The analysis further indicates that the price effects of consolidation differ across chains, suggesting that antitrust authorities should focus their attention more heavily on mergers involving firms whose consumers view outlets as close substitutes.

\section{Motivation and Background}

\subsection{Motivation}

The central motivation of Thomadsen’s paper is to understand how ownership structure and geographic differentiation jointly shape pricing in retail markets. Many consumer-facing industries such as fast food, retail, gasoline, and grocery are composed of firms that operate multiple outlets spread across local geographic areas as franchises. In these settings, changes in ownership structure, such as mergers between nearby outlets, may change competitive incentives by allowing firms to internalize price competition across the outlets they control.

This raises an important policy question: \textit{How do mergers in geographically differentiated industries affect equilibrium prices?} This is the question that Thomadsen aims to answer. Addressing this question matters for antitrust authorities such as the Federal Trade Commission (FTC), who evaluate mergers partly on the basis of whether they increase market power and harm consumers. Determining the price effects of multi-store ownership is therefore directly relevant for merger review and competition policy.

The contribution of this paper is to combine strands of the literature that had typically been studied separately. Previous research examined either (i) the role of geography in price competition or (ii) the implications of multi-store ownership, but few papers incorporated both elements in a structural framework. Thomadsen fills this gap by estimating a model of demand and firm pricing that explicitly accounts for spatial differentiation and ownership structure, allowing him to simulate how prices would change under different merger scenarios.

A key novelty of the paper is that it integrates spatial differentiation with multi-store ownership in a fully structural pricing model. Earlier empirical studies of spatial competition typically held ownership constant and focused on how geographic distance affects substitution patterns and price sensitivity. Similarly, previous work on multi-store ownership often abstracted from geography, treating markets as undifferentiated or relying on aggregate data. Thomadsen’s framework brings these two perspectives together by incorporating ownership structure directly into firms' profit maximization problems while also modeling how consumers substitute across geographically dispersed outlets. This allows the analysis to capture how a merger between nearby franchisees alters not only direct competition between two outlets, but also the broader pattern of consumer substitution. As a result, the paper can generate policy-relevant counterfactuals that earlier approaches were not able to evaluate.

\subsection{Background}

Thomadsen’s empirical setting is the fast food industry in Santa Clara County, California, focusing on McDonald’s and Burger King outlets. These chains are well-suited for this type of analysis for several reasons. First, they are two of the largest fast food brands in the United States and represent a substantial share of local consumption in the study area. Second, most outlets are franchised rather than corporately owned, creating variation in ownership structure across locations. Third, their menus are highly standardized, which helps isolate pricing incentives rather than product differences.

To construct the dataset, Thomadsen collected prices and outlet characteristics by personally visiting each restaurant. He recorded menu prices, documented outlet features, and mapped precise geographic coordinates. Corporate-owned stores and outlets located inside airports or military bases were excluded to avoid atypical pricing environments. He also focuses only on the price of the Big Mac and Whopper value meals, which are the most commonly purchased items at McDonald’s and Burger King. Some of the summary statistics for the sample are reported in Table~\ref{tab:summary}.

This setting provides a clean and realistic environment for studying how geography and ownership structure interact to determine equilibrium prices. The combination of detailed price data, geographic information, and natural variation in franchise ownership allows the structural model to estimate demand and simulate counterfactual pricing outcomes under alternative ownership.

\begin{table}[H]
\centering
\begin{threeparttable}
\caption{Summary Statistics.}
\label{tab:summary}
\small
\setlength{\tabcolsep}{4pt}

\begin{tabularx}{\textwidth}{Xccccc}
\toprule
\textbf{Variable} & \textbf{N} & \textbf{Mean} & \textbf{Std. Dev.} & \textbf{Min} & \textbf{Max} \\
\midrule
Burger King price                 & 38 & 3.26    & 0.11    & 3.19    & 3.69    \\
McDonald's price                  & 41 & 3.46    & 0.27    & 2.99    & 4.09    \\
Burger King dummy                 & 79 & 0.481   & 0.503   & 0       & 1       \\
Number BK/McD within 2 miles      & 79 & 3.835   & 1.918   & 1       & 8       \\
Number co-owned within 2 miles    & 79 & 0.532   & 0.765   & 0       & 2       \\
Number other hamburger within 2 miles & 79 & 4.443   & 1.831   & 0       & 9       \\
Distance to BK/McDonald's (miles) & 79 & 0.691   & 0.488   & 0.006   & 1.933   \\
Distance to co-owned (miles)      & 79 & 1.426   & 1.550   & 0       & 5       \\
Distance to other hamburger (miles)&79& 0.630   & 0.548   & 0.005   & 2.739   \\
Multiple-owner dummy              & 79 & 0.633   & 0.485   & 0       & 1       \\
Drive-thru dummy                  & 79 & 0.684   & 0.468   & 0       & 1       \\
Mall dummy                        & 79 & 0.089   & 0.286   & 0       & 1       \\
Population density (per mi$^2$)   & 79 & 6{,}454.6 & 4{,}251.4 & 11.4  & 20{,}964.6 \\
Worker density (per mi$^2$)       & 79 & 934.6   & 5{,}879.8 & 0      & 49{,}260.1 \\
Median income (\$)                & 79 & 46{,}250 & 7{,}811   & 28{,}750 & 67{,}500 \\
\bottomrule
\end{tabularx}

\begin{tablenotes}[flushleft]
\footnotesize
\item Note: Population and worker density are measured as (number of people)/(mile$^2$).
\end{tablenotes}

\end{threeparttable}
\end{table}

\section{Empirical Strategy and Theoretical Model}

Thomadsen first provides regressions that relate prices to ownership concentration, geographic proximity to competitors, and outlet characteristics. These regressions show clear patterns: prices are higher when nearby outlets share an owner and lower when competing outlets are located closer together. These findings provide important motivation by confirming that both ownership structure and geography matter for pricing behavior.

However, regressions have clear limitations. They summarize correlations but do not model the full spatial layout of the market, nor do they identify causal effects or describe firms’ strategic interactions. In particular, they cannot predict how prices would change under alternative ownership structures because they do not incorporate the firms’ profit maximization problems or the way ownership influences substitution across outlets. As a result, reduced-form regressions cannot be used on their own to evaluate merger policies or simulate counterfactual scenarios.

A structural model addresses these limitations by modeling consumer demand, spatial differentiation, and firms’ pricing decisions in a unified framework. This approach allows Thomadsen to uncover the mechanisms that drive pricing, such as substitution patterns, marginal costs, and the degree of spatial competition. Importantly, the structural model makes it possible to simulate hypothetical mergers and quantify their pricing effects. In doing so, it provides a more complete and policy-relevant tool for analyzing competitive outcomes than regressions alone.

\subsection{Reduced Form Model}

Thomadsen estimates a series of regressions to show how prices vary with local competition and ownership concentration. In his paper, he does not present the regression equations explicitly, but a representative specification consistent with Table~\ref{tab:regression} can be written as

\begin{equation}
\label{eq:regression}
\log(P_j) 
= 
\alpha 
+ \beta_1 \, \text{Comp}_j
+ \beta_2 \, \text{CoOwn}_j
+ \beta_3 X_j
+ \beta_4 \text{Demo}_j
+ \varepsilon_j,
\end{equation}

where 
$\text{Comp}_j$ measures the number of competing outlets within a two-mile radius,
$\text{CoOwn}_j$ measures the number of commonly owned outlets,
$X_j$ includes outlet characteristics, such as drive-through or mall locations, 
$\text{Demo}_j$ captures local demographic composition, 
and $\varepsilon_j$ is an idiosyncratic error term.

We can rewrite this in matrix form, similar to how it is often presented in econometrics and mathematical economics, as

\begin{equation}
\label{eq:ols}
\mathbf{y} = \mathbf{X}\beta + \varepsilon,
\end{equation}

where $\mathbf{y}$ is the vector of log prices across outlets, 
$\mathbf{X}$ contains the covariates described above,
and $\beta$ is the vector of regression coefficients. 
Then the OLS estimator is given by

\begin{equation}
\label{eq:ols_formula}
\hat{\beta} = (\mathbf{X}^\top \mathbf{X})^{-1} \mathbf{X}^\top \mathbf{y},
\end{equation}

which highlights the linear algebra underlying least squares estimation.

\subsection{Structural Model}

\subsubsection{Demand}

To move beyond correlations, Thomadsen models consumer choice explicitly. Conditional indirect utility for consumer $i$ at location $j$ is

\begin{equation}
V_{i,j} = X^{\prime}_j\beta - D_{i,j}\delta - P_{j}\gamma + \eta_{i,j},
\end{equation}

where $X^{\prime}_j$ is a vector indicating chain identity, the presence of a drive-through or play area, and whether the outlet is located in a mall. $D_{i,j}$ represents the distance between consumer $i$ and outlet $j$, and $P_j$ is the price of a meal at outlet $j$.

If a consumer eats elsewhere (outside McDonald's or Burger King), they receive utility

\begin{equation}
V_{i,0} = \beta_{0} + M_{i}\pi + \eta_{i,0},
\end{equation}

where $M_i$ is a vector of the consumer's age, gender, and race. The share of consumers at a particular location $b$ and demographic type $M$ who consume from outlet $j$ is given by

\begin{equation}
S_{j,b}(P, X, M \mid \beta, \delta, \gamma, \pi) = \int_{A_j} f(\eta_{i})\,d\eta_{i},
\end{equation}

where $P$ is the $J$-dimensional vector of prices for every outlet in the market, and

\begin{equation}
A_j = \{\eta_i \mid (V_{i,j} > V_{i,t} \;\forall t \neq j) \cap (V_{i,j} > V_{i,0})\}
\end{equation}

is the set of match values $\eta_i$ such that the consumer derives higher utility from consuming at outlet $j$ than from any other outlet or from the outside good.

Under the logit structure, the choice probability simplifies to

\begin{equation}
S_{j,b}(P, X, M \mid \beta, \delta, \gamma, \pi)
=
\frac{
    e^{X'_j \beta - D_{b,j} \delta - P_j \gamma}
}{
    e^{\pi M} + \sum_{t=1}^{J} e^{X'_t \beta - D_{b,t} \delta - P_t \gamma}
}.
\end{equation}

Total demand for each outlet is then calculated by summing the product of the choice probabilities and the mass of consumers of each demographic at each location:

\begin{equation}
Q_{j}(P,X \mid \beta, \delta, \gamma, \pi) = \sum_{b} \sum_{M} h(b,M)S_{j,b}(P, X, M \mid \beta, \delta, \gamma, \pi),
\end{equation}

where $h(b,M)$ is the number of consumers of type $M$ at location $b$.

\subsubsection{Supply}

On the supply side, assume that there are $F$ firms (franchisees), each owning a subset $F_f$ of the $j = 1,..., J$ outlets. Also assume firm costs consist of fixed costs and constant unit marginal cost. Profits for firm $f$ are then given by

\begin{equation}
\Pi_f = \sum_{j \in F_f} (r_{k} P_{j} Q_{j}(P) - c_{j} Q_{j}(P) - FC_j),
\end{equation}

where $FC_j$ is the fixed cost of operating outlet $j$, 
$c_j$ is the marginal cost of a meal at outlet $j$, 
$r_k$ is the fraction of revenue that franchisees of chain $k$ retain after paying royalties, 
and $P$ is the $J$-dimensional vector of prices for all outlets.

Rewriting in terms of effective marginal cost gives

\begin{equation}
\Pi_f = \sum_{j \in F_f}\left(P_j Q_{j}(P) - \left(\frac{c_j}{r_k}\right)Q_{j}(P) - \frac{FC_j}{r_k}\right),
\end{equation}

where 

\begin{equation}
C_j = \frac{c_j}{r_k}
\end{equation}

and

\begin{equation}
C_j = C_k + \varepsilon_j
\end{equation}

decomposes marginal cost into a chain-specific component $C_k$ and an outlet-specific cost shock $\varepsilon_j$.

The firm’s problem on the supply side is to maximize profits:

\begin{maxi*}|s|
{}{\Pi_f = \sum_{j \in F_f}\left(P_j Q_{j}(P) - \left(\frac{c_j}{r_k}\right)Q_{j}(P) - \frac{FC_j}{r_k}\right)}
{}{}
\end{maxi*}

This is the same type of unconstrained maximization problem that appears in standard microeconomic theory: firms choose prices to satisfy first-order conditions. For outlet $j$, the first-order condition under multi-store ownership can be written as

\begin{equation}
Q_{j}(P) = \sum_{r \in F_f} (P_{r} - C_{k} - \varepsilon_{r}) \frac{\partial Q_{r}(P)}{\partial P_j}.
\end{equation}

These $J$ equations can be solved for each $\varepsilon_{j}$. To do this, Thomadsen defines a matrix $\Omega$ as

\begin{equation}
\Omega_{j,r} = 
\begin{cases}
\displaystyle \frac{\partial Q_r}{\partial P_j}, & \text{if $r$ and $j$ have the same owner}, \\[8pt]
0, & \text{otherwise}.
\end{cases}
\end{equation}

This simplifies the system of first-order conditions to

\begin{equation}
Q(P) + \Omega(P - C - \varepsilon) = 0,
\end{equation}

where $Q(P)$, $C$, and $\varepsilon$ are the vectors of quantities, chain-specific marginal costs, and outlet-specific marginal costs respectively. 

Like many papers that analyze private firms, Thomadsen does not observe quantity data directly, nor can he compute derivatives of quantity with respect to price in closed form. Instead, he solves for them as functions of the model parameters:

\begin{equation}
Q(P, X \mid \theta) + \Omega(P, X \mid \theta)(P - C - \varepsilon) = 0.
\end{equation}

Rearranging gives an expression for the cost residuals:
\begin{equation}
\varepsilon = P - C + \Omega(P, X \mid \theta)^{-1} Q(P, X \mid \theta).
\end{equation}

This equation is central for estimation because $\varepsilon$ is what enters the Generalized Method of Moments (GMM) procedure.

\subsubsection{GMM Estimation}

The structural parameters $\theta$ are estimated using GMM. The key condition is that cost shocks should be uncorrelated with a set of instruments $Z_j$:

\begin{equation}
E[\varepsilon_{j}(\theta^{*}) \mid Z_j] = 0.
\end{equation}

The sample analog is

\begin{equation}
G_J(\theta) 
= \frac{1}{J} \sum_{j=1}^{J} Z_j \, \varepsilon_j(\theta).
\end{equation}

Each row in this system corresponds to a different demographic instrument, such as the fraction of nearby consumers in a particular age group, gender category, or racial group. Because these characteristics shift demand patterns but are exogenous to marginal costs, they provide variation that helps identify the structural parameters.

These moment conditions collectively determine the GMM objective function, which selects the parameter vector $\hat{\theta}$ that makes the empirical moments as close to zero as possible. Formally, the estimator solves

\begin{argmini}
{\hat{\theta}}{G_{J}(\theta)^{\prime} AG_{J}(\theta)}{}{}
\end{argmini}

for a positive definite weighting matrix $A$. Conceptually, we want to find the parameters so that the predicted cost shocks generated by the structural model are orthogonal to demographic variation. The moment conditions complete the bridge between the observed data, the theoretical model, and the equilibrium pricing behavior that the paper seeks to quantify.

\section{Results}

The regression estimates in Table~\ref{tab:regression} provide evidence that both 
ownership structure and geographic proximity shape equilibrium prices. Outlets located near a 
larger number of competing McDonald's or Burger King stores charge lower prices, which is consistent with standard competitive pressure. In contrast, outlets that have a greater number of 
commonly owned stores within two miles charge higher prices. Geographic distance reinforces these patterns: being closer to a competitor lowers prices, while being closer to a co-owned outlet raises them. These results establish the empirical motivation for the structural model by showing that both competition and ownership concentration have economically meaningful effects on pricing.

The structural estimates reinforce the regression findings by quantifying the substitution patterns 
that drive pricing incentives. The demand parameters imply that consumers are highly sensitive to 
geographic distance, showing that spatial differentiation is a key source of market power. The estimated price sensitivity parameter shows that consumers respond strongly to price differences across 
outlets, amplifying the effect of multi-store ownership. If two nearby outlets share an owner, that 
owner internalizes the substitution between them and raises prices accordingly. 

The estimated marginal costs vary across chains and outlet characteristics. The cost shocks ($\varepsilon_j$) are centered near zero when evaluated at the estimated parameters, indicating that the model and instruments satisfy the GMM moment conditions.

\subsection*{Merger Simulations and Counterfactual Experiments}

Thomadsen conducts several experiments that examine how prices change when previously independent franchisees merge.

\paragraph{Mergers between nearby outlets}
When two nearby outlets merge, equilibrium prices increase substantially. Because consumers view nearby outlets as close substitutes, a merged owner has the ability to raise prices at one store without losing as many customers to the other. This is the mechanism through which multi-store ownership raises prices.

\paragraph{Mergers between distant outlets}
Surprisingly, mergers also raise prices even when outlets are far enough apart that they would not 
compete strongly under separate ownership. This occurs because changing ownership structure 
changes the entire pricing equilibrium. Even distant stores contribute to the profit maximization 
problem of the merged firm. This result shows that the effects of mergers extend beyond 
immediate geographic neighborhoods and cannot be captured by simple distance rules.

\paragraph{Chain-level differences}
The simulations show that mergers among McDonald's lead to much larger price increases 
than mergers among Burger King outlets. This reflects different substitution patterns and marginal costs 
across chains.

\begin{table}[H]
\centering
\begin{threeparttable}
\caption{Descriptive Regression Results. \\ \small Dependent Variable: $\log(\text{Price})$.}
\label{tab:regression}
\small

\begin{tabularx}{0.9\textwidth}{Xcc}
\toprule
\textbf{Variable} & \textbf{Coefficient} & \textbf{Std. Error} \\
\midrule
Burger King dummy                   & $0.048^{**}$  & (0.013) \\
Number BK/McD within 2 miles        & $-0.035^{*}$  & (0.004) \\
Number co-owned within 2 miles      & $0.025^{**}$  & (0.009) \\
Distance to BK/McD (miles)          & $0.028^{*}$   & (0.014) \\
Distance to co-owned (miles)        & $-0.012^{*}$  & (0.006) \\
Drive-thru dummy                    & 0.014         & (0.014) \\
Population density (per mi$^2$)     & $0.396^{**}$  & (0.167) \\
Worker density (per mi$^2$)         & $-0.343^{*}$  & (0.143) \\
\midrule
$R^2$                               & 0.513         &         \\
\bottomrule
\end{tabularx}

\begin{tablenotes}[flushleft]
\footnotesize
\item Standard errors in parentheses. Statistical significance:
\textsuperscript{***} $p<0.01$, 
\textsuperscript{**} $p<0.05$, 
\textsuperscript{*} $p<0.10$.
\end{tablenotes}

\end{threeparttable}
\end{table}

\section{Discussion}

Thomadsen’s paper has many strengths. First, he provides one of the earliest structural models that combines spatial differentiation with multi-store ownership, offering a richer understanding of pricing behavior in franchise markets. The paper demonstrates how regressions and structural methods complement one another: the regressions establish empirical patterns, while the structural model clarifies the mechanisms behind those patterns. Second, the paper’s counterfactual simulations give the analysis direct policy relevance. By showing that mergers raise prices even when outlets are geographically distant, his results highlight the importance of looking beyond simple distance thresholds in merger review. Third, the model makes good use of limited data by leveraging demographic instruments, spatial structure, and equilibrium conditions.

The paper also has limitations. One limitation is that consumers choose between only two fast food chains and the outside option. In reality, consumers face a much wider menu of alternatives, and omitting these options may overstate substitution between  
McDonald’s and Burger King. Additionally, the model assumes that marginal costs are constant within 
each chain. This could overlook important cost differences, such as differences in labor markets, input prices, or management quality. Another limitation is the cross-sectional nature of the data. Without variation over time, it is difficult to separate structural relationships from unobserved local characteristics.

Future research could address these limitations in several ways. First, incorporating additional fast food chains or sit-down restaurants would provide a more realistic representation of consumer choices. Second, collecting panel data would allow for strategies that control for unobserved heterogeneity, including store fixed effects. Third, researchers could examine dynamic pricing incentives, which are left out of the static model. Extensions could also explore how online ordering or delivery services alter competition and modify the results of the paper.

Overall, while the model is necessarily simplified, it provides a powerful framework for studying how 
mergers in differentiated product markets affect equilibrium outcomes.

\section{Conclusion}

Thomadsen’s paper answers an important question in industrial organization: how does ownership structure interact with geographic differentiation to shape equilibrium prices? Using regressions and a structural model of demand and supply, the paper shows that multi-store ownership increases prices and that the magnitude of these effects depends on how closely consumers view outlets as substitutes. The simulations reveal that mergers influence pricing even across outlets that do not directly compete under separate ownership.

These findings have direct implications for antitrust policy. Approaches that define 
markets using simple geographic rules may underestimate the consequences of mergers. By integrating 
theoretical structure with empirical data, the paper offers a better framework for evaluating competitive effects and provides a foundation for more sophisticated analysis.

The mathematics used in the paper align closely with standard tools in mathematical economics. The firm's pricing problem is formulated as an unconstrained maximization problem, solved through first-order conditions. The vectors of prices, quantities, and marginal costs mirror the vector notation used in linear algebra, and the ownership matrix $\Omega$ is a direct application of matrix methods to represent firm decisions across multiple outlets. Even the GMM estimator relies on minimizing a quadratic form in the moment conditions, which is structurally similar to least squares and other optimization problems.

Overall, the paper illustrates how the mathematical tools of optimization and linear algebra can be combined to answer economically meaningful questions. Thomadsen's study shows the value of structural methods in applied microeconomics and demonstrates how economic modeling can provide insights that go beyond reduced-form patterns in the data. The paper contributes both methodologically and substantively to our understanding of pricing in differentiated product markets.

\nocite{*}
\bibliographystyle{apalike}
\bibliography{references}

\end{document}