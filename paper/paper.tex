% Preamble
\documentclass[11pt, letterpaper]{article}
\usepackage{newtxtext}
\usepackage{newtxmath}
\title{Reflection Paper -- Mathematical Economics}
\author{Joe White}
\date{\today}

\begin{document}
\maketitle
\newpage

% Start Section 1: Covers the motivation as well as background of the paper
\section{Motivation and Background}
\subsection{Motivation}
The motivation behind this paper is to understand how ownership structure, geography, and prices are all linked together. It was very important to understand this relationship as market power, structure, and mergers are all important towards creating competitive environments. The research question of this paper is how do mergers in differentiated industries differ in their effect on prices. 

% Start Section 2: Outlines the paper and explains what the paper does, the papers models, and results
\section{Overview}
\subsection{What does the paper accomplish?}
\subsection{Methodology}
\subsection{Results}

\section{Model}
\subsection{Demand}
\begin{equation}
V_{i,j} = X^{\prime}\beta - D_{i,j}\delta - P_{j}\gamma + \eta_{i,j}
\end{equation}

\begin{equation}
V_{i,0} = \beta_{0} + M_{i}\pi + \eta_{i,0}
\end{equation}

\begin{equation}
S_{i,b}(P, X, M \mid \beta, \delta, \gamma, \pi) = \int_{A_j} f(\eta_{i})\,d\eta_{i}
\end{equation}

\begin{equation}
A_j = \{\eta_i \mid (V_{i,j} > V_{i,t} \;\forall t \neq j) \cap (V_{i,j} > V_{i,0})\}
\end{equation}

\begin{equation}
S_{j,b}(P, X, M \mid \beta, \delta, \gamma, \pi)
=
\frac{
    e^{X'_j \beta - D_{b,j} \delta - P_j \gamma}
}{
    e^{\pi M} + \sum_{t=1}^{J} e^{X'_t \beta - D_{b,t} \delta - P_t \gamma}
}.
\end{equation}

\begin{equation}
Q_{j}(P,X \mid \beta, \delta, \gamma, \pi) = \sum_{b} \sum_{M} h(b,M)S_{i,b}(P, X, M \mid \beta, \delta, \gamma, \pi)
\end{equation}

\begin{equation}
\frac{\partial Q_{j}(P,X \mid \beta, \delta, \gamma, \pi)}{\partial P_k} = \frac{\partial \sum_{b} \sum_{M} h(b,M)S_{i,b}(P, X, M \mid \beta, \delta, \gamma, \pi)}{\partial P_k}
\end{equation}

\subsection{Supply}

\begin{equation}
\Pi_f = \sum_{j \in F_f} (r_{k} P_{j} Q_{j}(P) - c_{j} Q_{j}(P) - FC_j)
\end{equation}

\begin{equation}
\Pi_f = \sum_{j \in F_f}(P_j Q_{j}(P) - (\frac{c_j}{r_k})Q_{j}(P) - \frac{FC_j}{r_k})
\end{equation}

\begin{equation}
C_j = (C_k + \varepsilon_j)
\end{equation}

\begin{equation}
Q_{j}(P) = \sum_{r \in F_f} (P_{r} - C_{k} - \varepsilon_{r}) \frac{\partial Q_{r}(P)}{\partial P_j} = 0
\end{equation}

\begin{equation}
\Omega_{j,r} = 
\begin{cases}
\displaystyle \frac{\partial Q_r}{\partial P_j}, & \text{if $r$ and $j$ have the same owner}, \\[8pt]
0, & \text{otherwise}.
\end{cases}
\end{equation}

\begin{equation}
Q(P) + \Omega(P - C - \varepsilon) = 0
\end{equation}

\begin{equation}
Q(P, X \mid \theta) + \Omega(P, X \mid \theta)(P - C - \varepsilon) = 0
\end{equation}

\begin{equation}
\varepsilon = P - C + \Omega(P, X \mid \theta)^{-1} Q(P, X \mid \theta)
\end{equation}

\begin{equation}
E[\varepsilon_{j}(\theta^{*}) \mid Z_j] = 0
\end{equation}

\begin{subequations}
\label{}

\begin{align}
G_J(\theta) 
&= \frac{1}{J} \sum_{j=1}^{J} Z_j \, \varepsilon_j(\theta)
\label{}
\\[8pt]
G_J(\theta) 
&= \frac{1}{J} \sum_{j=1}^{J}
    \left[
        R_{18\text{-}29}
        \frac{Q_j(M_{\text{Age}},\theta)}{\text{Pop}(M_{\text{Age}})}
        \;-\;
        R_{\text{Age}}
        \frac{Q_j(M_{18\text{-}29},\theta)}{\text{Pop}(M_{18\text{-}29})}
    \right]
\label{}
\\[8pt]
G_J(\theta) 
&= \frac{1}{J} \sum_{j=1}^{J}
    \left[
        R_{\text{Male}}
        \frac{Q_j(M_{\text{Female}},\theta)}{\text{Pop}(M_{\text{Female}})}
        \;-\;
        R_{\text{Female}}
        \frac{Q_j(M_{\text{Male}},\theta)}{\text{Pop}(M_{\text{Male}})}
    \right]
\label{}
\\[8pt]
G_J(\theta) 
&= \frac{1}{J} \sum_{j=1}^{J}
    \left[
        R_{\text{Black}}
        \frac{Q_j(M_{\text{White}},\theta)}{\text{Pop}(M_{\text{White}})}
        \;-\;
        R_{\text{White}}
        \frac{Q_j(M_{\text{Black}},\theta)}{\text{Pop}(M_{\text{Black}})}
    \right]
\label{}
\end{align}

\end{subequations}


\end{document}
